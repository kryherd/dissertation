\documentclass[../dissertation.tex]{subfiles}
 
\begin{document}

\section{General Introduction}

	Categories help us organize the world. They help us predict and hypothesize about category members, helping us quickly select the most appropriate response for each situation. We also rely on language during these processes. As \citet{Lupyan2012} puts it, language augments our thought. For categories, language provides structure in the form of category labels, but language also affects how we think about and even perceive the categories themselves. Thus any thorough investigation of how we learn categories must consider the role of language. \par
	Indeed, many theoretical frameworks of category learning involve language -- some even reference it right in the name. For example, a key theory in perceptual category learning, COVIS, stands for "competition between verbal and implicit systems'" \citep{Ashby1998}. Similarly, a theory put forth by Minda and colleagues is called "A theory of verbal and nonverbal category learning" \citep{Minda2010}. However, to date most theories of category learning that consider language primarily determine whether language has an influence on systems for category learning, rather than further defining the role language plays in these systems.\par
	Thus, the current work seeks to both define a theory of category learning and explore the role language has in this theory. In this review I will synthesize multiple approaches to category learning, all of which have some type of dual-systems model. Following the synthesis, I will review relevant literature that provides suggestions as to how language might be involved in category learning in a dual-systems model. Through these efforts, I will provide a theoretical framework and hypotheses for this dissertation.

\subsection{Dual-systems model for category learning}
	Multiple theories converge on the idea that there are two systems for category learning. In this section, I will first describe a generalized dual-systems model that pulls threads from all of these theories and then go on to describe how each theory fits into the overarching framework. 
	
\subsubsection{Proposed model}
	The proposed model involves two systems for category learning. The first, which I title the \textbf{associative system}, uses associative mechanisms in an iterative manner to learn distributions of features. This system is best suited for learning multidimensional \textit{similarity-based} categories such as natural kinds, where it is difficult to describe necessary and sufficient rules for inclusion. Similarity-based categories have features that are correlated and probabilistic, such that a category instance may not have all of the category-relevant features but tends to have some distribution of them. For example, Manx cats do not have tails, a typical feature of cats, but are still undeniably members of the category "cat." Thus, the associative system must be able to extract the most frequent pattern of features over many instances in order to learn a category.
\subsubsection{COVIS}
\subsubsection{Dimensionality}
\subsubsection{Statistical Density}
\subsubsection{Verbal/nonverbal}
\subsubsection{Taxonomic/thematic}

 
\subsection{Vocabulary/labels and category learning}

\subsection{Executive function and category learning}

\end{document}

