\documentclass[../dissertation.tex]{subfiles}
 
\begin{document}

\section{General Discussion}

\subsection{Language in a dual-systems model: summary of results}
\subsubsection{Language and the interaction between two systems}
In Experiment 1, I tested whether the order in which an individual engages different category learning systems affects learning in those systems. I also tested whether any observed order effects vary according to general language ability. I predicted that switching from the hypothesis-testing system to the associative system would produce a cost, while switching from associative to hypothesis-testing would not. This prediction was based on prior research showing dominance of the hypothesis-testing system \citep{Erickson2008, Ashby2010}. I also predicted that switch costs would be higher in individuals with poorer language ability, since my prior research showed that low-language individuals had difficulty switching away from suboptimal learning strategies \citep{Ryherd2019}. \par 
Neither of these hypotheses were supported by the data. Instead of poorer performance in second blocks, which would indicate a switch cost, all order effects were driven by better performance in second blocks, a result more consistent with a learning effect. In addition, there was no interaction between language ability and order in any of the three analyses. Thus, the observed learning effect did not depend on an individual's language ability.

\subsubsection{Individual differences and dual-systems category learning}

\subsubsection{Levels of processing and the dual-systems model}

The second analysis in Experiment 2 was the first to directly compare performance across three different paradigms designed to test dual-systems approaches to category learning. These approaches have considerable theoretical similarities (see Chapter \ref{intro}); as such, it is conceivable that the experimental tasks used to test these approaches measure the same type of processing. I tested subjects on the three tasks and hypothesized that the pattern of performance would be similar across all three. That is, if participants showed higher accuracy on the associative block than the hypothesis-testing block of one task, they should show the same pattern for the other two tasks. \par
This hypothesis was not supported by the results. In accuracy, participants showed significant differences between the associative and hypothesis-testing blocks for the Ashby perceptual category learning task, but no block differences for the Sloutsky statistical density task or the taxonomic/thematic task. In reaction time, the Ashby and Sloutsky tasks showed similar patterns, while the taxonomic/thematic task was the reverse. Together, the results suggest that these three tasks as they are typically administered are not comparable and do not engage the same processing.



\subsection{Rethinking the dual-systems model of categorization}

\subsubsection{Existing critiques of dual-systems models}

While many studies have shown dissociations between the two category learning systems proposed in COVIS, a recent line of research has brought this framework into question based on three main claims. First, it suggests that common stimuli used in a COVIS paradigm are not sufficiently matched, and thus double dissociations seen in these paradigms are due to stimulus characteristics rather than differential processing. For example, one study tested the effect of feedback using stimuli that were matched on participant error rates, category separation, and relevant dimensions \citep{Edmunds2015}. This study did not find a different between category type, a finding in opposition to previous studies with less carefully-matched stimuli \citep{Ashby2002, Maddox2003}. \par 
	Another critique of the COVIS framework is its assumption that items learned by the associative system are learned nonverbally and implicitly, and thus are not available to the conscious mind. To test this assumption, another study tested recognition memory for exemplars of rule-based (hypothesis-testing) and information-integration (associative) stimuli after the categories were learned \citep{Edmunds2016}. Recognition memory is commonly assumed to test explicit memory \citep{Gabrieli1995}. If participants could reliably recognize exemplars from information-integration categories, it would be unlikely that these items were being learned truly implicitly. In fact, \citet{Edmunds2016} found that participants not only were able to recognize information-integration (associative) stimuli at an above-chance rate, they were also more accurate at recognizing information-integration (associative) stimuli than rule-based (hypothesis-testing) stimuli. This suggests that instances that should have been learned using the associative system implicitly were at least available to explicit memory after learning. Further support for this critique comes from a study showing that participants produced verbal reports of their learning strategies that matched model-based strategy determination \cite{Edmunds2015}. Thus, participants were able to access both the items they had learned as well as the method they had learned for categorization.  \par 
	The third critique of COVIS centers on the previously-mentioned mathematical models, which are used to verify whether an individual is using associative or hypothesis-testing strategies to learn categories. Decision-bound strategy analysis fits different decision boundary models to the category responses made by participants to determine their learning strategies \citep{Maddox1993}. 


\end{document}

