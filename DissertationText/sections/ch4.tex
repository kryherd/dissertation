\documentclass[../dissertation.tex]{subfiles}
 
\begin{document}

\section{General Discussion}

\subsection{Language in a dual-systems model: summary of results}
\subsubsection{Language and the interaction between two systems}
In Experiment 1, I tested whether the order in which an individual engages different category learning systems affects learning in those systems. I also tested whether any observed order effects vary according to general language ability. I predicted that switching from the hypothesis-testing system to the associative system would produce a cost, while switching from associative to hypothesis-testing would not. This prediction was based on prior research showing dominance of the hypothesis-testing system \citep{Erickson2008, Ashby2010}. I also predicted that switch costs would be higher in individuals with poorer language ability, since my prior research showed that low-language individuals had difficulty switching away from suboptimal learning strategies \citep{Ryherd2019}. \par 
Neither of these hypotheses were supported by the data. Instead of poorer performance in second blocks, which would indicate a switch cost, all order effects were driven by better performance in second blocks, a result more consistent with a learning effect. In addition, there was no interaction between language ability and order in any of the three analyses. Thus, the observed learning effect did not depend on an individual's language ability.

\subsubsection{Individual differences and dual-systems category learning}

\subsubsection{Levels of processing and the dual-systems model}

The second analysis in Experiment 2 was the first to directly compare performance across three different paradigms designed to test dual-systems approaches to category learning. These approaches have considerable theoretical similarities (see Chapter \ref{intro}); as such, it is conceivable that the experimental tasks used to test these approaches measure the same type of processing. I tested subjects on the three tasks and hypothesized that the pattern of performance would be similar across all three. That is, if participants showed higher accuracy on the associative block than the hypothesis-testing block of one task, they should show the same pattern for the other two tasks. \par
This hypothesis was not supported by the results. In accuracy, participants showed significant differences between the associative and hypothesis-testing blocks for the Ashby perceptual category learning task, but no block differences for the Sloutsky statistical density task or the taxonomic/thematic task. In reaction time, the Ashby and Sloutsky tasks showed similar patterns, while the taxonomic/thematic task was the reverse. Together, the results suggest that these three tasks as they are typically administered are not comparable and do not engage the same processing.



\subsection{Rethinking the dual-systems model of categorization}


\end{document}

