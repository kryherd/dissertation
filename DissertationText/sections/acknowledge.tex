\documentclass[../dissertation.tex]{subfiles}
 
\begin{document}
\begin{center} Acknowledgments
\end{center}
	I have to start by thanking my advisor, Nicole Landi. She has supported my graduate career since before it began; as a rising senior in an REU, I was advised to send out interest emails to faculty I might want to work with. She was one of only a few that I worked up the nerve to email, and she quickly engaged in a conversation with me about research ideas and tips for applying to UConn that lead me here. Throughout my time as a graduate student, she has been a Goldilocks advisor -- not too hands-off, not a micromanager; brought me into new research projects, but not too many; compassionate but practical. I owe most of my success to her guidance and flexibility; as my own expertise grew, her advising style shifted. I could probably write 50 more pages about the ways Nicole has supported me. I will always be grateful to her and I really treasure the work we've been able to do together. \par 
	I also want to be sure to acknowledge the entire language science community at UConn. Both the faculty and the students have made my experience at UConn truly broad and deep. I feel lucky to have been trained by experts in so many inter-related fields and I truly believe that UConn will be known as \textit{the} place for language in the coming years. I've worked with so many people in my different capacities at UConn -- as a TA, as a student, as a researcher, as an IBRAIN fellow, as a Langfest or Open House organizer -- I don't feel that I can thank everyone who has shaped my experience. I can say that I feel blessed to be surrounded with so many role models to look up to. \par 
	In terms of peers, I can't overstate the benefits of having a great labmate. Meaghan Perdue is a research rockstar and an all-around great person. It's so great to have someone to talk to about lab stuff and life stuff, who really gets what it is like to live in Bousfield A302. Having spent two years on my own in the lab, I can say for certain that Meaghan improved the lab in many ways. Kara, if you ever see this, please cherish your time with her! \par 
	At this point, I need to acknowledge the non-academics who have made my graduate career what it is. My family has supported me the whole way -- through the ups and the (very significant) downs. My New Haven community is so strong, and I am endlessly grateful for the network of people I've managed to find down here. Nina Gumkowski, Olivia Harold, and Eliezer Weinbach kept me sane for the two years I lived by Storrs. My other squad members and drinks night pals made moving to New Haven feel like coming home. And, of course, my boyfriend Ryan Petersburg has helped me grow more in the past two and half years than I probably did in the five years before that. \par 
	It wouldn't be something written by me if I didn't mention my cat. Paris was my first priority after I moved to Storrs. I got a paycheck, then I got a cat. When I lived in Willington in a mostly empty apartment, she was my primary companion. She has been the only stable element of my graduate career -- she does the same thing every time I come home and she curls up in a tiny ball the same way she always has. She loved blankets four years ago, and she still loves them now. \par 
	Finally, I would like to thank all of the people who directly contributed to this document. I'd like to thank my committee for guiding me through the process of developing and executing a research plan. I want to thank my undergraduate RAs for running \textit{so many} subjects. I want to thank the undergrads that provided their data. I also want to make sure to throw significant thanks to Charles Davis, my collaborator on a lot of the research included in this document. I also want to thank you -- if you've made it through this long acknowledgments section, you might just make it through the actual dissertation.
\end{document}
