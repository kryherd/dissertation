\documentclass[../dissertation.tex]{subfiles}
 
\begin{document}
\begin{center}
\ptitle
\linebreak
Kayleigh Ryherd, PhD
\linebreak
University of Connecticut, 2019
\end{center}
Multiple theories of category learning converge on the idea that there are two systems for categorization, each designed to process different types of category structures. The associative system learns categories that have probabilistic boundaries and multiple overlapping features through iterative association of features and feedback. The hypothesis-testing system learns rule-based categories through explicit testing of hypotheses about category boundaries. Prior research suggests that language resources are necessary for the hypothesis-testing system but not for the associative system. However, other research emphasizes the role of verbal labels in learning the probabilistic similarity-based categories best learned by the associative system. This suggests that language may be relevant for the associative system in a different way than it is relevant for the hypothesis-testing system. Thus, this study investigated the ways in which language plays a role in the two systems for category learning. In the first experiment, I tested whether language is related to an individual's ability to switch between the associative and hypothesis-testing systems. I found that participants showed remarkable ability to switch between systems regardless of their language ability. The second experiment directly compared three dual-systems approaches to category learning and tested whether individual differences in language-related skills like vocabulary and executive function were related to category learning performance. This experiment showed different patterns of performance for each category learning approach despite considerable theoretical overlap. It also showed that performance in each approach was related to different individual difference measures. I conclude by questioning the applicability of a dual-systems model to all levels of processing and discuss ways in which future research can further elucidate the role of language in category learning for categories of different structures.
\end{document}

