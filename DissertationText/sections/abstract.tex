\documentclass[../dissertation.tex]{subfiles}
 
\begin{document}
\begin{center}
\ptitle
\linebreak
Kayleigh Ryherd, PhD
\linebreak
University of Connecticut, 2019
\end{center}
Multiple theories of category learning converge on the idea that there are two systems for categorization, each designed to process different types of category structures. The associative system learns categories that have probabilistic boundaries and multiple overlapping features through iterative association of features and feedback. The hypothesis-testing system learns rule-based categories through explicit testing of hypotheses about category boundaries. This study investigated the ways in which language plays a role in these two systems for category learning. In the first experiment, I test whether language is related to an individual's ability to switch between systems. I found that participants show remarkable ability to switch between systems regardless of their language ability. The second experiment directly compares three dual-systems approaches to category learning and tests whether individual differences in language-related skills like vocabulary and executive function are related to category learning performance. This experiment shows that despite considerable theoretical overlap between these approaches, each approach reveals a different pattern of performance as well as varying relationships to vocabulary and executive function. I conclude by questioning the applicability of a dual-systems model to all levels of processing and discuss ways in which future research can further elucidate the role of language in category learning for categories of different structures.
\end{document}

